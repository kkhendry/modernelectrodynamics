\documentclass{article}
\usepackage{amsmath}
\title{Mathematical Preliminaries}\label{c1}
\author{Amey Joshi}
\date{28-Jan-2024}
\begin{document}
\maketitle
\begin{enumerate}
\item[1.1] Levi-Civita tensor.
\begin{enumerate}
\item[(a)] $\hat{e}_1, \hat{e}_2, \hat{e}_3$ are unit vectors along three 
Cartesian axes. Show that
\[
\epsilon_{ijk} = \hat{e}_i\cdot(\hat{e}_j \times \hat{e}_k).
\]
If $j = k$ or $i = j$ the triple product vanishes. The magnitude of the
triple product is $\pm 1$. Its sign depends on whether $i, j, k$ are in
even or odd permutation. Thus, the properties of rhs are the same as that of
the Levi-Civita tensor.

\item[(b)] Quite straightforward.

\item[(c)] Show that $\epsilon_{ijk}\epsilon_{ist} = \delta_{js}\delta_{kt} 
- \delta_{jt}\delta_{ks}$.
We start from lhs,
\begin{equation}\label{e1}
\epsilon_{ijk}\epsilon_{ist} = \epsilon_{1jk}\epsilon_{1st} +
                              \epsilon_{2jk}\epsilon_{2st} +
                              \epsilon_{3jk}\epsilon_{3st} 
\end{equation}
In the first term on the rhs, $j, k$ can be either $2, 3$ or $3, 2$ for it
to be non-zero. Thus, there are four possibilities:
\begin{itemize}
\item $j = 2, k = 3, s = 2, t = 3$ when the term is $1$.
\item $j = 3, k = 2, s = 2, t = 3$ when the term is $-1$.
\item $j = 3, k = 2, s = 3, t = 2$ when the term is $1$.
\item $j = 2, k = 3, s = 3, t = 2$ when the term is $-1$.
\end{itemize}
These are also the possible values of $\delta_{js}\delta_{kt} - \delta_{jt}
\delta_{ks}$. Further, when $j, k$ take these values, the other terms on
the rhs of equation \eqref{e1} are zero.

Identical analysis can be used to evaluate the remaining terms.

\item[(d)] Given that $\vec{a}$ and $\vec{b}$ are constant vectors. If
$\vec{L}^{op}$ is the angular momentum operator then show that
\[
[\vec{L}^{op}\cdot\vec{a}, \vec{L}^{op}\cdot\vec{b}] = i\hbar\vec{L}^{op}
\cdot (\vec{a} \times \vec{b}).
\]
We start with the left hand side,
\begin{eqnarray*}
[\vec{L}^{op}\cdot\vec{a}, \vec{L}^{op}\cdot\vec{b}] 
    &=& [L^{op}_i a_i, L^{op}_j b_j] \\
    &=& L^{op}_i a_i L^{op}_j b_j - L^{op}_j b_i L^{op}_i a_i \\
    &=& a_ib_j(L^{op}_i L^{op}_j - L^{op}_j L^{op}_i) \\
    &=& a_ib_j[L^{op}_i, L^{op}_j] \\
    &=& a_ib_j\epsilon_{ijk}L^{op}_k \\
    &=& i\hbar\vec{L}^{op}\cdot(\vec{a} \times \vec{b}).
\end{eqnarray*}
\end{enumerate}

\item[1.2] Levi-Civita tensor.
\begin{enumerate}
\item[(a)] $\delta_{ii} = \delta_{11} + \delta_{22} + \delta_{33} = 3$.
\item[(b)] $\delta_{ij}\epsilon_{ijk} = 0$ because when $i = j$ the first
term is $1$ but the second term is $0$ and when $i \ne j$ it is the other
way round.
\item[(c)] $\epsilon_{ijk}\epsilon_{ljk} = \delta_{il}$. Note that $j$ and
$k$ must be different for the two terms on the lhs to be non-zero. That
leaves only one possibility for both $i$ and $l$. Further, the product will
be $1$ irrespective of the sign of the individual terms because they are
identical when they are non-zero.
\item[(d)] $\epsilon_{ijk}\epsilon_{ijk} = 1$.
\end{enumerate}

\item[1.3] Vector identities.
\begin{enumerate}
\item[(a)] $\vec{A} \times \vec{B} = \epsilon_{ijk}\hat{e}_iA_jB_k$ and
$\vec{C} \times \vec{D} = \epsilon_{lmn}\hat{e}_lC_mD_n$ so that
$(\vec{A} \times \vec{B})\cdot(\vec{C} \times \vec{D}) = \epsilon_{ijk}
\hat{e}_iA_jB_k \cdot \epsilon_{lmn}\hat{e}_lC_mD_n = \epsilon_{ijk}
\epsilon_{lmn}A_jB_kC_mD_n\hat{e}_i\cdot\hat{e}_l = \epsilon_{ijk}
\epsilon_{lmn}A_jB_kC_mD_n\delta_{il} = \epsilon_{ijk}\epsilon_{imn}
A_jB_kC_mD_n = (\delta_{jm}\delta_{kn} - \delta_{jn}\delta_{km})A_jB_kC_m
D_n = A_jC_j B_kD_k - A_jD_j - B_kC_k = \vec{A}\cdot\vec{C}\vec{B}\cdot
\vec{D} - \vec{A}\cdot\vec{D}\vec{B}\cdot\vec{C}$.

\item[(b)] \begin{eqnarray*}
\nabla\cdot(\vec{f}\times\vec{g}) &=& (\hat{e}_l\partial_l)
\cdot(\epsilon_{ijk}\hat{e}_if_jg_k) \\
 &=& \delta_{il}\partial_l\epsilon_{ijk}f_jg_k \\
 &=& \epsilon_{ijk}\partial_i(f_jg_k) \\
 &=& \epsilon_{ijk}(g_k\partial_if_j + f_j\partial_ig_k) \\
 &=& g_k\epsilon_{ijk}\partial_if_j + f_j\epsilon_{ijk}\partial_i g_k \\
 &=& g_k\epsilon_{kij}\partial_i f_j + f_j\epsilon_{jki}\partial_i g_k \\
 &=& g_k\epsilon_{kij}\partial_if_j - f_j\epsilon_{jik}\partial_i g_k \\
 &=&\vec{g}\cdot\nabla\times\vec{f} - \vec{f}\cdot\nabla\times\vec{g}
\end{eqnarray*}

\item[(c)] Let $\vec{A} \times \vec{B} = \vec{E}$ and $\vec{C} \times 
\vec{D} = \vec{F}$ so that  $(\vec{A} \times \vec{B}) \times (\vec{C} 
\times \vec{D}) = \vec{E} \times \vec{F} = \epsilon_{ijk}\hat{e}_iE_jF_k
= \epsilon_{ijk}\hat{e}_i\epsilon_{jlm}A_lB_m\epsilon_{krs}C_rD_s$. Now,
\begin{eqnarray*}
\epsilon_{ijk}\epsilon_{jlm} &=& -\epsilon_{jki}\epsilon_{jlm} \\
 &=& -(\delta_{kl}\delta_{im} - \delta_{km}\delta_{li}) \\
 &=& \delta_{km}\delta_{li} - \delta_{kl}\delta_{im}
\end{eqnarray*}
so that 
\begin{eqnarray*}
\vec{E} \times \vec{F} &=& lB_m\epsilon_{krs}C_rD_s \\
 &=& \hat{e}_i(A_iB_k\epsilon_{krs}C_rD_S - A_kB_i\epsilon_{krs}C_rD_s\\
 &=& (\epsilon_{krs}B_kC_rD_S)A_i\hat{e}_i - 
     (\epsilon_{krs}A_kC_rD_s)B_i\hat{e}_i \\
 &=& (\vec{B}\cdot(\vec{C} \times \vec{D}))\vec{A} - 
     (\vec{A}\cdot(\vec{C} \times \vec{D}))\vec{B}.
\end{eqnarray*}

\item[(d)] Show that $(\vec{\sigma}\cdot\vec{a})(\vec{\sigma}\cdot\vec{b}) = 
\vec{a}\cdot\vec{b} + i\vec{\sigma}\cdot(\vec{a}\times\vec{b})$.
$(\vec{\sigma}\cdot\vec{a})(\vec{\sigma}\cdot\vec{b}) = \sigma_i a_i \sigma_jb_j
= a_ib_j\sigma_i\sigma_j = a_ib_j(\delta_{ij} + i\epsilon_{ijk}\sigma_k)
= a_ib_i + i\epsilon_{ijk}a_ib_j\sigma_k = \vec{a}\cdot\vec{b} + i\epsilon_{kij}
\sigma_ka_ib_j = \vec{a}\cdot\vec{b} + i\vec{\sigma}\cdot(\vec{a}\times\vec{b})$.
\end{enumerate}

\item[(1.4)] Vector derivative identities.
\begin{enumerate}
\item[(a)] $\nabla\cdot(f\vec{g}) = \hat{e}_i\partial_i(f\hat{e}_jg_j) =
\hat{e}_i\cdot\hat{e}_j\partial_i(fg_j) = \delta_{ij}\partial_i(fg_j) =
\partial_i(fg_i) = \nabla f \cdot\vec{g} + f\nabla\cdot\vec{g}$.

\item[(b)] $\nabla\times(f\vec{g}) = \hat{e}_i\epsilon_{ijk}\partial_j(fg_k)
= \hat{e}_i\epsilon_{ijk}(\partial_j f)g_k + \hat{e}_i\epsilon_{ijk}f
(\partial_jg_k) = \nabla f\times \vec{g} + f\nabla\times\vec{g} = 
f\nabla\times\vec{g} - \vec{g}\times\nabla f$.

\item[(c)] 
\begin{eqnarray*}
\nabla\times(\vec{g}\times\vec{r}) &=& 
    \hat{e}_i\epsilon_{ijk}\partial_j(\epsilon_{klm}g_lr_m) \\
 &=& \hat{e}_i\epsilon_{ijk}\epsilon_{klm}\partial_j(g_lr_m) \\
 &=& \hat{e}_i\epsilon_{kij}\epsilon_{klm}\partial_j(g_lr_m) \\
 &=& \hat{e}_i(\delta_{il}\delta_{jm} - \delta_{im}\delta_{jl})
     \partial_j(g_lr_m) \\
 &=& \hat{e}_i(\partial_j(g_ir_j) - \partial_j(g_jr_i)) \\
 &=& \hat{e}_i(r_j\partial_j g_i + g_i\partial_jr_j) -
     \hat{e}_i(r_i\partial_jg_j + g_j\partial_jr_i) \\
 &=& \vec{r}\cdot\nabla\vec{g} + 3\vec{g} - \vec{r}\nabla\cdot{g} 
     - \hat{e}_ig_j\delta_{ij} \\
 &=& \vec{r}\cdot\nabla\vec{g} + 3\vec{g} - \vec{r}\nabla\cdot{g} - \vec{g} \\
 &=& \vec{r}\cdot\nabla\vec{g} + 2\vec{g} - \vec{r}\nabla\cdot{g}
\end{eqnarray*}
\end{enumerate}

\item[(1.11)] Some integral identities.
\begin{enumerate}
\item[(a)] Since $\nabla\cdot\vec{G} = 0$, $\nabla\cdot(\phi\vec{G}) = 
\nabla\phi\cdot\vec{G} + \phi\nabla\cdot\vec{G} = \nabla\phi\cdot\vec{G}$. Thus,
\begin{eqnarray*}
\int d^3r \vec{F}\cdot\vec{G} &=& \int d^3r \nabla\phi\cdot\vec{G} \\
 &=& \int d^3r \nabla\cdot(\phi\vec{G})  \\
 &=& \int d^2r \hat{n}\cdot \phi\vec{G}
\end{eqnarray*}
If we choose the volume of integration sufficiently large so that the surface
integral is evaluated over a surface very far away from the origin. The area
of the surface goes as $O(r^2)$ but the integrand goes as $O(1/r^{2 + \alpha})$
for some $\alpha > 0$. Therefore, the surface integral vanishes as $r 
\rightarrow \infty$.

\item[(b)] Since $\nabla\times\vec{G} = 0$, $\nabla\times(\phi\vec{G}) = 
\nabla\phi\times\vec{G} + \phi\nabla\times\vec{G} = \nabla\phi\times\vec{G}$. 
In equation (1.81) of the book, let $\vec{P}$ be a constant and $\vec{R} =
\nabla\times\vec{Q}$. Then we get
\[
\int_V d^3r (-\vec{P}\cdot\nabla\times\vec{R}) = 
\int_Sd\vec{S}\cdot(\vec{P} \times \vec{R})
\]
Therefore,
\[
-\vec{P}\cdot\int_V d^3r \nabla\times\vec{R} = 
\int_S\vec{P}\cdot(\vec{R}\times d\vec{S})
\]
or
\[
-\vec{P}\cdot\int_V d^3r \nabla\times\vec{R} = 
-\vec{P}\cdot\int_S d\vec{S} \times \vec{R}
\]
or, since $\vec{P}$ is arbitrary,
\begin{equation}\label{e2}
\int_V d^3r \nabla\times\vec{R} = \int_S d\vec{S} \times \vec{R}.
\end{equation}
If $\vec{R} = \phi\vec{G}$ then as $\nabla\times\vec{G} = 0$,
\[
\int_V d^3r\nabla\phi\times\vec{G} = \int_S d\vec{S} \times (\phi\vec{G}).
\]
The integral on the left vanishes for the same reason as the previous case.

\item[(c)] We first show that 
\begin{equation}\label{e3}
\int_V d^3r \partial_j(P_j\vec{G}) = \int_S dS\hat{n}\cdot\vec{P}\vec{G}.
\end{equation}
Expanding the integral on the left hand side,
\begin{eqnarray*}
\int_V d^3r \partial_j(P_j\vec{G}) &=& \int_V dx_1dx_2dx_3\partial_1(P_1\vec{G}) + \\
 & & \int_V dx_1dx_2dx_3\partial_2(P_2\vec{G}) + \\
 & & \int_V dx_1dx_2dx_3\partial_3(P_3\vec{G}) + \\
 &=& \int_Sdx_2dx_3 P_1\vec{G} + \int_Sdx_1dx_3 P_2\vec{G} + \\
 & & \int_Sdx_1dx_2 P_3\vec{G} \\
 &=& \int_S dS\hat{n}\cdot\vec{P}\vec{G}.
\end{eqnarray*}
We also have
\[
\partial_i(P_i\vec{G}) = (\nabla\cdot\vec{P})\vec{G} + \vec{P}\cdot\nabla\vec{G}.
\]
Therefore,
\[
\int_V d^3r \left((\nabla\cdot\vec{P})\vec{G} + \vec{P}\cdot\nabla\vec{G}\right) =
\int_S dS\hat{n}\cdot\vec{P}\vec{G}.
\]
If $\vec{G} = \vec{r}$, as $\vec{P}\cdot\nabla\vec{r} = \vec{P}$, we get
\[
\int_V d^3r (\nabla\cdot\vec{P})\vec{r} + \int_V d^3r \vec{P} =
\int_S dS\hat{n}\cdot\vec{P}\vec{r}.
\]
\end{enumerate}

\item[(1.16)] We use equation (1.100) of the book, 
\[
\delta(f(x)) = \sum_{n}\frac{\delta(x - x_n)}{|f^\prime(x_n)|},
\]
where $f(x_n) = 0$ and $f^\prime(x_n) \ne 0$.
\begin{enumerate}
\item[(a)] 
\[
g_1(E) = \int_{-\infty}^\infty dk_x\delta(E - k_x^2).
\]
Here, $f(k_x) = E - k_x^2$ whose zeros are $\pm\sqrt{E}$ so that
\[
g_1(E) = \int_{-\infty}^\infty dk_x\left(\frac{\delta(k_x - \sqrt{E})}{|-2\sqrt{E}|}
+ \frac{\delta(k_x + \sqrt{E})}{|2\sqrt{E}|}\right) = \frac{1}{\sqrt{E}}.
\]

\item[(b)] 
\begin{eqnarray*}
g_2(E) &=& \int_{-\infty}^\infty dk_xdk_y\delta(E - k_x^2 - k_y^2) \\
 &=& 2\pi\int_0^\infty dk \delta(E - k^2)k 
\end{eqnarray*}
Here, $f(k_x) = E - k^2$ whose zeros are $\pm\sqrt{E}$ so that
\[
g_2(E) = 2\pi\int_0^\infty dk \left(\frac{k\delta(k - \sqrt{E})}{|-2\sqrt{E}|}
+ \frac{k\delta(k + \sqrt{E})}{|2\sqrt{E}|}\right) = 2\pi\frac{1}{2} = \pi.
\]

\item[(c)] 
\begin{eqnarray*}
g_3(E) &=& \int_{-\infty}^\infty dk_xdk_y\delta(E - k_x^2 - k_y^2 - k_z^2) \\
 &=& 4\pi\int_0^\infty dk \delta(E - k^2)k^2 
\end{eqnarray*}
Here, $f(k) = E - k^2$ whose zeros are $\pm\sqrt{E}$ so that
\begin{eqnarray*}
g_3(E) &=& 4\pi\int_0^\infty dk \left(\frac{k^2\delta(k - \sqrt{E})}{|-2\sqrt{E}|}
+ \frac{k^2\delta(k + \sqrt{E})}{|2\sqrt{E}|}\right) \\
 &=& 4\pi\frac{\sqrt{E}}{2} \\
 &=& 2\pi\sqrt{E}.
\end{eqnarray*}
\end{enumerate}

\item[(1.17)] Dot and cross products.
\begin{enumerate}
\item[(a)] $\vec{b} \times \hat{n} = \epsilon_{klm}\hat{e}_kb_ln_m$, so that
\begin{eqnarray*}
\hat{n}\times (\vec{b} \times \hat{n}) 
   &=& \epsilon_{ijk}\hat{e}_in_j\epsilon_{klm}b_ln_m \\
   &=& \epsilon_{kij}\epsilon_{klm}n_jb_ln_m \\
   &=& (\delta_{il}\delta_{jm} - \delta_{im}\delta_{jl})\hat{e}_in_jb_ln_m \\
   &=& \hat{e}_in_jb_in_j - \hat{e}_in_jb_jn_i \\
   &=& \vec{b} - \hat{n}(\vec{b}\cdot\hat{n})
\end{eqnarray*}.

This identity can be proved with fewer steps by using the BAC-CAB rule.

\item[(b)] $\vec{b}\cdot\hat{n}$ is the projection of $\vec{b}$ along $\hat{n}$
and $\hat{n} \times (\vec{b} \times \hat{n})$ is the projection in a direction 
perpendicular to $\hat{n}$.

\item[(c)] $\vec{B} \times \vec{C} = \omega^{-2}(\vec{c} \times \vec{a}) \times
(\vec{a} \times \vec{b}) = \omega^{-2}(\vec{a}(\vec{c} \times \vec{a})\cdot\vec{b} -
\vec{b}(\vec{c} \times \vec{a})\cdot\vec{a}) = \omega^{-2}(\omega - 0) = 
\omega^{-1}$.

If $\vec{a}, \vec{b}, \vec{c}$ are the vectors in direct space then $\vec{A},
\vec{B}, \vec{C}$ are the vectors in the reciprocal space.
\end{enumerate}

\item[(1.18)] $S_{ij}$ and $T_{ij}$.
\begin{enumerate}
\item[(a)] $\epsilon_{ijk}S_{ij} = 0$ is $S_{ij} - S_{ji} = 0$, that is, when the
tensor $S_{ab}$ is symmetric.
\item[(b)] The components of $\vec{y}$ are
\begin{eqnarray*}
y_1 &=& b_1T_{11} + b_2T_{21} + b_3T_{31} \\
    &=& b_2T_{21} - b_3(-T_{31}) \\
y_2 &=& b_1T_{12} + b_2T_{22} + b_3T_{32} \\
    &=& b_3T_{32} - b_1(-T_{12}) \\
y_3 &=& b_1T_{13} + b_2T_{23} + b_3T_{33} \\
    &=& b_1T_{13} - b_2(-T_{23})
\end{eqnarray*}
If $\vec{\omega} = (-T_{23}, -T_{31}, -T_{12})$ then $\vec{y} = \vec{b} \times 
\vec{\omega}$. In terms of the components of $\vec{\omega}$, the anti-symmetric
tensor $T_{ab}$ is
\[
\begin{bmatrix}
0 & -\omega_3 & \omega_2 \\
\omega_1 & 0 & -\omega_1 \\
-\omega_2 & \omega_1 & 0
\end{bmatrix}.
\]
\end{enumerate}

\item[(1.19)] Two surface integrals.
\begin{enumerate}
\item[(a)] A physicist's ``proof": If we move along the surface, for every 
element $d\vec{S}$ there is another one pointing in the opposite direction 
because we are integrating over a closed surface.

A better proof follows from the identity
\[
\int_S d\vec{S}\phi = \int_V d^3r\nabla\phi
\]
by putting $\phi = 1$. The identity itself can be proved as
\[
\int_S d\vec{S}\cdot(\vec{P}\phi) = \int_V \nabla\cdot(\vec{P}\phi).
\]
If $\vec{P}$ is a constant vector,
\[
\vec{P}\cdot \int_S d\vec{S}\phi  = \vec{P}\cdot\int_V d^3r\nabla\phi.
\]
Since $\vec{P}$ is arbitrary, the identity follows.

\item[(b)] 
\[
\frac{1}{3}\int_S d\vec{S}\cdot\vec{r} = \frac{1}{3}\int_V d^3r 
\nabla\cdot\vec{r} = \frac{1}{3}\int_V d^3r 3 = V.
\]
\end{enumerate}

\item[(1.20)] The potential is $\varphi(\vec{r}) = (\vec{a} \times \vec{r})
\cdot (\vec{b} \times \vec{r})$, where $\vec{a}$ and $\vec{b}$ are constant
vectors. We can simplify it as
\[
\varphi(\vec{r}) = \vec{a}\cdot(\vec{r} \times (\vec{b} \times \vec{r})) = 
\vec{a}\cdot(\vec{b}r^2 - \vec{r}(\vec{r}\cdot\vec{b})) = \vec{a}\cdot\vec{b}
r^2 - (\vec{a}\cdot\vec{r})(\vec{b}\cdot\vec{r}).
\]
Now,
\[
\nabla r^2 = \hat{e}_i\partial_i r^2 = 2\vec{r}
\]
and
\[
\nabla ((\vec{a}\cdot\vec{r})(\vec{b}\cdot\vec{r})) = (\vec{b}\cdot\vec{r})
\nabla (\vec{a}\cdot\vec{r}) + (\vec{a}\cdot\vec{r})\nabla(\vec{b}\cdot\vec{r})
= \vec{a}(\vec{b}\cdot\vec{r}) + \vec{b}(\vec{a}\cdot\vec{r}). 
\]
Therefore,
\[
\vec{E} = -(2\vec{r} + \vec{a}(\vec{b}\cdot\vec{r}) + \vec{b}(\vec{a}\cdot\vec{r})).
\]
To compute charge density, we need
\[
\nabla\cdot(\vec{a}(\vec{b}\cdot\vec{r})) = 
\vec{a}\cdot\nabla(\vec{b}\cdot\vec{r}) = \vec{a}\cdot\vec{b}.
\]
Thus,
\[
\rho = \epsilon_0\nabla\cdot\vec{E} = -\epsilon_0(6 + 2\vec{a}\cdot\vec{b}) = 
-2\epsilon_0(3 + \vec{a}\cdot\vec{b}).
\]

\item[(1.21)] A decomposition identity.
\[
A_iB_j = \frac{A_iB_j + A_iB_j}{2} + \frac{A_iB_j - A_iB_j}{2}
= \frac{A_iB_j + A_iB_j}{2} + \frac{1}{2}\epsilon_{ijk}(\vec{A}\times\vec{B})_k.
\]
\end{enumerate}
\end{document}

