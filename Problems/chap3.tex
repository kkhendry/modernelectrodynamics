\documentclass{article}
\usepackage{amsmath, xcolor}
\newcommand{\un}{\hat{n}}
\newcommand{\uv}[1]{\hat{e}_{#1}}
\newcommand{\grad}[1]{\nabla{#1}}
\newcommand{\dive}[1]{\nabla\cdot\vec{#1}}
\newcommand{\curl}[1]{\nabla\times\vec{#1}}
\newcommand{\op}{\prime}
\newcommand{\td}[2]{\frac{d{#1}}{d{#2}}}
\newcommand{\pdt}[2]{\frac{\partial{#1}}{\partial{#2}}}
\newcommand{\ke}{\frac{1}{4\pi\epsilon_0}}
\title{Electrostatics}\label{c3}
\author{Amey Joshi}
\date{10-Mar-2024}
\begin{document}
\maketitle
\begin{enumerate}
\item[(1)] The potentials are $V_1$ and $V_2$ in the left and right half-planes
respectively. Therefore, the electric field is in the $x$ direction. It is non-
zero only along the $y$-axis where the potential changes. Since $\vec{E} = 
-\grad\varphi$,
\[
\vec{E} = -\uv{x}\lim_{\delta x \rightarrow 0}\frac{V_2 - V_1}{\delta}.
\]
As $V_2 < V_1$, the field is directed along the positive $x$ axis. A positively
charged particle will accelerate along the positive $x$-axis. Only the $x$-
component of the momentum changes, the $y$-component remains unchanged. If $\vec{p}$
and $\vec{p}^{\;\op}$ are the momenta in negative and positive half-planes, $p_x^\op = 
p_x + qE$, $p_y^\op = p_y$, $p = \sqrt{p_x^2 + p_y^2}, p^\op = \sqrt{{p_x^\op}^2
+ {p_y^\op}^2}$, $p_y = p\sin\theta_1, p_y^\op = p^\op\sin\theta_2$. Thus,
\[
(p_x^2 + p_y^2)\sin^2\theta_1 = ({p_x^\op}^2 + {p_y^\op}^2)\sin^2\theta_2,
\]
or
\[
(p_x^2 + p_y^2)\sin^2\theta_1 = ((p_x + qE)^2 + p_y^2)\sin^2\theta_2.
\]
Thus,
\begin{equation}\label{e1}
\frac{\sin\theta_1}{\sin\theta_2} = 
\sqrt{1 + \frac{2p_xqE + q^2E^2}{p_x^2 + p_y^2}} =
\sqrt{1 + \frac{2\cos\theta_1 qE}{p} + \frac{q^2E^2}{p^2}}
\end{equation}
The ratio of the sines of the two angles is not independent of the ``angle of
incidence'' because the amount of ``refraction'' depends on how much of the
total momentum is affected by the field.

A step field of this kind described in this problem shows up at the semiconductor
pn junction.

\item[(2)] If $E_k = C_k + D_{kj}r_j$,
\[
\dive{E} = \dive{C} + \frac{\partial}{\partial x_k}(D_{kj}x_j) = \dive{C} +
x_j\pdt{D_{kj}}{x_j} + D_{kj}\pdt{x_j}{x_k},
\]
that is,
\[
\dive{E} = \dive{C} + D_{kk}.
\]
If $\vec{C}$ is a constant vector, then $\dive{E} = D_{kk}$. In a charge free
region, $\dive{E} = 0$. Therefore, the tensor $\underline{D}$ is traceless.

We also have $\curl{E} = 0$. Thus,
\[
\epsilon_{ijk}\partial_jE_k = \epsilon_{ijk}\partial_jC_k + \epsilon_{ijk}\partial_j(D_{kl}x_l)
= 0 + \epsilon_{ijk}D_{kl}\partial_j(x_l) = \epsilon_{ijk}D_{kl}\delta_{jl}.
\]
Thus, we have $\epsilon_{ijk}D_{jk} = 0$, that is, $D_{jk} = D_{kj}$.

If $\varphi = -C_kx_k - x_kD_{kj}x_j$ then $E_k = \partial_k\varphi = C_k + D_{kj}r_j$.

\item[(3a)] I'll quickly derive an expression for the electric field due to a
uniformly charged ring at a point on its axis. Let the ring be in the $xy$ plane
and its centre coincident with the origin. At any point on the $z$ axis only the
$z$ component of the field will survive. Consider an element $dl$ of the ring
with charge $\lambda dl$. If $r$ if the radius of the ring then its contribution
to the field at a point $(0, 0, z)$ will be
\[
dE = \ke\frac{\lambda dl}{r^2 + z^2}\cos\theta,
\]
where $\theta$ is the polar angle. It is easy to see that 
\[
\cos\theta = \frac{z}{\sqrt{r^2 + z^2}}
\]
so that
\[
dE = \ke\frac{z\lambda dl}{(r^2 + z^2)^{3/2}}
\]
Since $dl = rd\phi$,
\begin{equation}\label{e2}
\vec{E} = \frac{1}{2\epsilon_0}\frac{\lambda rz}{(r^2 + z^2)^{3/2}}\uv{z}.
\end{equation}
If $q = 2\pi r \lambda$,
\begin{equation}\label{e3}
\vec{E} = \ke\frac{qz}{(r^2 + z^2)^{3/2}}\uv{z}.
\end{equation}
We can get to the same result by first finding the potential. The potential
$d\varphi$ due to an element $dl$ is
\[
d\varphi = \ke\frac{\lambda dl}{\sqrt{r^2 + z^2}} = \ke\frac{\lambda r d\phi}{\sqrt{r^2 + z^2}},
\]
so that
\begin{equation}\label{e4}
\phi = \ke \frac{2\pi r\lambda }{\sqrt{r^2 + z^2}} = \ke \frac{q}{\sqrt{r^2 + z^2}}
\end{equation}
so that
\[
\vec{E} = -\grad\varphi = \ke\frac{qz}{(r^2 + z^2)^{3/2}}\uv{z}.
\]

Calculating the potential is much simpler than the field. We will, therefore, use
equation \eqref{e4} to calculate the potential due to a spherical shell at a point 
outside it. We can always orient the $z$ axis along the centre of the shell and the
field point. A shell of 
radius $a$ can be considered to be made up of rings of infinitesimal thickness whose
radii $b$ increase from $0$ to $a$ and then decrease to $0$ as the distance from the
centre of the ring to the field point increases from $z - a$ to $z$ to $z + a$. The
potential due to the elementary ring as a position $z^\op$ is
\[
d\varphi = \ke \frac{dq}{\sqrt{b^2 + (z - z^\op)^2}}.
\]
Now $b^2 + {z^\op}^2 = a^2$ so that
\[
d\varphi = \ke \frac{dq}{\sqrt{a^2 + z^2 - 2z^\op z}}.
\]
We can also write $b = a\sin\psi, z^\op = a\cos\psi$ so that as $\psi$ varies from
$0$ to $\pi$ the entire shell is generated. Thus,
\[
d\varphi = \ke \frac{dq}{\sqrt{a^2 + z^2 - 2az\cos\psi}}.
\]
We also have $dq = \sigma (2\pi b ad\psi)$ so that
\[
d\varphi = \frac{2\pi a^2 \sigma}{4\pi\epsilon_0} \frac{\sin\psi d\psi}{\sqrt{a^2 + z^2 - 2az\cos\psi}}.
\]
Therefore,
\[
\varphi = \int_0^\pi \frac{2\pi a^2 \sigma}{4\pi\epsilon_0} \frac{\sin\psi d\psi}{\sqrt{a^2 + z^2 - 2az\cos\psi}}.
\]
Let $u = a^2 + z^2 - 2az\cos\psi$ so that $du = 2az\sin\psi d\psi$ and the 
limits of integration are from $(z - a)^2$ to $(z + a)^2$. Thus,
\[
\varphi = \frac{2\pi a^2 \sigma}{4\pi\epsilon_0}\int_{(z-a)^2}^{(z+a)^2}\frac{1}{2az} \frac{du}{\sqrt{u}}
= \frac{2\pi a^2 \sigma}{4\pi\epsilon_0}\frac{1}{2az} (2\sqrt{u})\Big|_{(z-a)^2}^{(z+a)^2} = 
\frac{2\pi a^2 \sigma}{4\pi\epsilon_0}\frac{4a}{2az}
\]
Thus,
\begin{equation}\label{e5}
\varphi = \ke \frac{4\pi a^2\sigma}{z} = \ke \frac{q}{z}.
\end{equation}
The electric field is
\begin{equation}\label{e6}
\vec{E} = \ke \frac{q}{z^2}\uv{z}.
\end{equation}

The electric field inside a spherical shell is zero. It is an immediate consequence
of Gauss' law
\[
\oint_S\vec{E}\cdot\un da = \frac{Q_{\text{enc}}}{\epsilon_0},
\]
where $Q_{\text{enc}}$ is the charge enclosed by $S$. We choose $S$ to be a small
spherical surface around the field point, entirely inside the shell.

\item[(3b)] We will use the expression for potential due to a ring to calculate 
that due to a charged disc at a field point on the disc's axis. If $dq$ is the 
charge on an infinitesimally thick ring of radius $r$ and thickness $dr$ then 
$dq = \sigma \times 2\pi rdr$. From equation \eqref{e5},
\[
d\varphi = \ke\frac{dq}{z} = \ke\frac{2\pi\sigma rdr}{z}
\]
so that for a disc of radius $a$,
\[
\varphi = \int_0^a \ke\frac{2\pi\sigma rdr}{z} = \ke\frac{2\pi\sigma}{z}\frac{a^2}{2}.
\]
Since $\sigma\pi a^2 = q$, the total charge on the disc, we have
\begin{equation}\label{e7}
\varphi = \ke\frac{q}{z}
\end{equation}
and hence 
\begin{equation}\label{e8}
\vec{E} = \ke \frac{q}{z^2}\uv{z}.
\end{equation}
The calculation of the electric field at a point outside the sphere proceeds in
exactly the same way as in the case of the shell. For the case of the field point 
inside the sphere, we can either use Gauss' law or ``summation method''. To
demonstrate the latter method, let the field point be at a distance $r_0$ from the
centre of the sphere. The potential at the field point will get a contribution
from shells of all radii from $0$ to $r_0$. If $dr$ is the thickness of the shell
then its volume is $4\pi r^2dr$. If $\rho$ is the charge density then the total
charge on the shell is $4\pi r^2\rho dr$. From equation \eqref{e5}, it contributes
a potential
\[
d\varphi = \ke \frac{4\pi r^2\rho dr}{r_0}
\]
Potential due to all shells will be
\[
\varphi = \ke\frac{4\pi\rho}{r_0}\frac{r_0^3}{3} = \ke\frac{Q_{\text{enc}}}{r_0}
\]
The field derived from this potential is the same as derived from Gauss' law.

\item[4] We need Gauss' law in integral form:
\[
\oint\vec{E}\cdot\un da = \frac{1}{\epsilon_0}\int_V \rho dv.
\]
If the charge distribution is radial then $\vec{E}$ too has a radial symmetry,
\begin{enumerate}
\item[(a)] This is a one dimensional problem with charge density symmetric about
the origin. To find the electric field at $x = R$ we just need,
\[
\int_{-R}^R \rho dx = -\frac{\rho_0}{\kappa}(e^{-\kappa R} - e^{\kappa R})
= \frac{2\rho_0}{\kappa}\sinh(R).
\]
The surface integral is just $2E(R)$. Therefore,
\[
E(R) = \frac{2\rho_0}{\kappa\epsilon_0}\sinh(R).
\]

\item[(b)] In two dimensions, a spherical ``surface'' is just a circle of radius
$r$ and centre at the origin. The ``surface integral'' evaluates to $2\pi RE(R)$.
The charge enclosed in the surface is
\[
\iint_S e^{-kr}rdrd\theta = 
2\rho_0\pi\left(\frac{1}{\kappa^2} - \frac{(\kappa R + 1)e^{-\kappa R}}{\kappa^2}\right)
\]
so that
\[
E(R) = \frac{\rho_0}{R\kappa^2\epsilon_0}\left(1 - (\kappa R + 1)e^{-\kappa R}\right).
\]

\item[(c)] In three dimensions, the surface integral is $4\pi R^2E(R)$ and the 
charge enclosed by the surface is
\begin{eqnarray*}
\iiint_S \rho_0 e^{-\kappa r}r^2\sin\theta drd\theta d\phi
&=& 4\pi\rho_0 \int_0^R r^2 e^{-\kappa r}dr \\
&=& \frac{4\pi\rho_0}{\kappa^3}\left(2 - (2 + 2\kappa R + \kappa^2R^2)e^{-\kappa R}\right),
\end{eqnarray*}
so that
\[
E(R) = \frac{\rho_0}{R^2\epsilon_0\kappa^3}\left(2 - (2 + 2\kappa R + \kappa^2R^2)e^{-\kappa R}\right)
\]
\end{enumerate}

\item[(6)] The force on a volume element $\delta v$ of $\rho(\vec{r})$ is 
$\delta\vec{F} = \rho(\vec{r})\vec{E}(\vec{r})\delta v$. Therefore, the torque 
on it is
\[
\delta\vec{N} = \vec{r} \times \delta\vec{F} = \vec{r}\times\rho(\vec{r})\vec{E}(\vec{r})\delta v
\]
so that
\[
\vec{N} = \int d^r \vec{r}\times\rho(\vec{r})\vec{E}(\vec{r}).
\]
The result follows immediately from 
\[
\vec{E}(\vec{r}) = \ke\int d^3r^\op \frac{\rho(\vec{r})(\vec{r} - \vec{r}^{\;\op})}{|\vec{r} - \vec{r}^\op|^3}.
\]

\item[(9)] Let $S$ be a parallelpiped bounded by the planes $z = 0, z = a$ and 
infinite in all other directions. Then, using Gauss' law
\[
\int_S d\vec{S}\cdot\vec{E} = \frac{Q}{\epsilon_0}.
\]
A parallelpiped has six faces. Of these four that are perpendicular to the $z = 0$
plane are at an infinite distance from the charge but they are not themselves 
infinitely large. Therefore, the flux through them will be zero. That leaves us 
with
\[
\int_S d\vec{S}\cdot\vec{E} = \int_{z=0} d\vec{S}\cdot\vec{E} + \int_{z=-a}
 d\vec{S}\cdot\vec{E}
\]
By symmetric, the two integrals will contribute equally, so that
\[
\int_S d\vec{S}\cdot\vec{E} = 2\int_{z=0} d\vec{S}\cdot\vec{E}
\]
from which we get
\[
\int_{z=0} d\vec{S}\cdot\vec{E} = \frac{Q}{2\epsilon_0}.
\]


\item[(10)] Green's reciprocity theorem to prove a property of harmonic functions.
\begin{enumerate}
\item[(a)] Choose $f = \varphi, g = 1/r$ so that the Green's reciprocity theorem gives
\[
\int d^3 r\left(\varphi\nabla^2\left(\frac{1}{r}\right) - \left(\frac{1}{r}\right)\nabla^2\varphi\right)
= \int d\vec{S}\cdot\left(\varphi\grad{\left(\frac{1}{r}\right)} - \left(\frac{1}{r}\right)\grad\varphi\right).
\]
In a charge free region, $\nabla^2\varphi = 0$. Further, $\nabla^2(r^{-1}) = 
-4\pi\delta(0)$ so that the left hand side is
\[
-4\pi\varphi(0) = \int d\vec{S}\cdot\left(\varphi\left(-\frac{\vec{r}}{r^3}\right) +
\left(\frac{1}{r}\right)\vec{E}\right).
\]
For the integral on rhs choose a surface of a sphere of radius $r$ so that
\[
-4\pi\varphi(0) = -\frac{1}{r^2}\int d\vec{S}\cdot\varphi(r)\uv{r} +
\int d\vec{S}\cdot\vec{E}
\]
The second term on the rhs is zero because there is no charge enclosed within the
surface. Thus we have,
\[
\frac{1}{4\pi r^2}\int_0^\pi \sin\theta d\theta\int_0^{2\pi}d\phi\varphi(r) = \varphi(0)
\]
The expression on lhs is the average of $\varphi$ on a sphere of radius $r$.

\item[(b)] Consider a test charge $q$ in an electrostatic field produced by static
charges. A position $\vec{r}_0$ of $q$ is a stable point if the net field at $\vec{r}_0$
is zero. It is stable equilibrium if the potential at it is a minimum. For sake of
notational convenience, shift the origin to $\vec{r}_0$. If suppose it were a minimum 
and we consider its average on a small sphere centred around it with radius $a$ then
by the equation
\[
\frac{1}{4\pi a^2}\int_0^\pi \sin\theta d\theta\int_0^{2\pi}d\phi\varphi() = \varphi(0).
\]
This means that there are some points on the surface at which the potential has
a value lower than it has at the origin. Therefore, the origin cannot be a stable
equilibrium.
\end{enumerate}

\item[(11)] {\color{red}I accidentally solved this problem using Gaussian units.}
Electrostatics of a charged disc. Let the disc be in the $xy$ plane with
its centre at the origin so that its axis of symmetry coincides with the $z$-axis.
\begin{enumerate}
\item[(a)] At any point on the $z$ axis, consider the contribution to $\varphi$ due
to an area element $rdrd\phi$ on the disc. It is
\[
d\varphi(z) = \frac{\sigma rdrd\phi}{\sqrt{z^2 + r^2}}
\]
so that
\[
\varphi(z) = \int_0^a\int_0^{2\pi}\frac{\sigma rdrd\phi}{\sqrt{z^2 + r^2}}
= 2\pi\sigma\int_0^a \frac{rdr}{\sqrt{z^2 + r^2}}.
\]
Put $u^2 = z^2 + r^2$ so that $udu = rdr$ and the limits of integration go
from $z$ to $\sqrt{z^2 + a^2}$ and
\[
\varphi(z) = 2\pi\sigma\int_z^{\sqrt{z^2 + a^2}} du = 2\pi\sigma(\sqrt{z^2 + a^2} - z)
\]
Now, 
\[
\sqrt{z^2 + a^2} = z\sqrt{1 + \frac{a^2}{z^2}} = z\left(1 + \frac{1}{2}\frac{a^2}{z^2}
- \frac{1}{8}\frac{a^4}{z^4} + \cdots\right) = z + \frac{1}{2}\frac{a^2}{z}
- \frac{1}{8}\frac{a^4}{z^3} + \cdots
\]
so that
\[
\varphi(z) = 2\pi\sigma\left(\frac{1}{2}\frac{a^2}{z} - \frac{1}{8}\frac{a^4}{z^3}
 + \cdots\right)
\]
Although the expression for $\varphi$ appeared alarming at a first glance, it is 
not so.

The electric field is
\[
E_z = \td{\varphi}{z} = 2\pi\sigma\left(\frac{1}{2}\frac{2z}{\sqrt{z^2 + a^2}} - 1\right)
\]
or
\[
\vec{E} = 2\pi\sigma\left(\frac{z}{\sqrt{z^2 + a^2}} - 1\right)\uv{z}.
\]
If $q$ is the total charge on the disc, $\sigma = q/(\pi a^2)$ so that
\[
\vec{E} = \frac{2q}{a^2}\left(\frac{z}{\sqrt{z^2 + a^2}} - 1\right)\uv{z}.
\]
Now,
\[
\lim_{z \rightarrow \infty}\vec{E} = \uv{z}\frac{2q}{a^2}\lim_{z \rightarrow \infty}
\left(\frac{1}{\sqrt{1 + a^2/z^2}} - 1\right) = 0.
\]

\item[(b)] A lack of symmetry makes this problem difficult. Move the origin to
the point on the rim where potential is calculated. Align the $x$ axis so that
the diameter from the origin is along it. The disc above the $x$ axis contributes
equally as the one from below it. So let us focus only on the half-disc above
the $x$ axis. Consider a line from $O$ making an angle $\theta$ with the $x$
axis. The distance of a point on this line ranges from $0$ to $2a\cos\theta$.
The contribution to an element on this line is
\[
d\varphi = \frac{\sigma rdrd\theta}{r}.
\]
The limits of integration are $\theta = 0$ to $\theta = \pi/2$ and for each
$\theta$, $r$ from $0$ to $2a\cos\theta$. Thus,
\[
\phi = \int_0^{\pi/2}\int_0^{2a\cos\theta}\sigma dr d\theta = 2a\sigma
\int_0^{\pi/2} \cos\theta d\theta = 2a\sigma
\]
\end{enumerate}

\item[12] We consider the remaining portion of the shell to be made up of 
elementary rings centred along the line joining the centre of the shell and $P$.
If $dq$ is the charge on one such ring then, if we align the $z$ axis so that
the origin coincides with the centre of the shell and $P$ lies on it, then
\[
dE = \ke\frac{dq z}{(r^2 + z^2)^{3/2}},
\]
where $r$ is the radius of the ring and $z$ is the distance of $P$ from the 
ring's centre. In terms of the radius $a$ of the shell, $r = a\sin\theta$,
$z = a - a\cos\theta$. The shell without the cap can be constructed by letting
$\theta$ range from $\theta_0$ to $\pi$. Thus,
\[
E = \ke \int_{\theta_0}^\pi \frac{a(1 - \cos\theta) dq}{(a^2\sin^2\theta + (a - a\cos\theta)^2)^{3/2}}
= \ke \frac{1}{4a^2}\int_{\theta_0}^\pi \frac{dq}{\sin(\theta/2)}.
\]
Now, $dq = \sigma \times 2\pi r \times a d\theta = 2\pi a^2\sin\theta d\theta = 
4\pi a^2\sin(\theta/2)\cos(\theta/2)$ so that
\[
E = \frac{\sigma}{4\epsilon_0}\int_{\theta_0}^\pi\cos\left(\frac{\theta}{2}\right)d\theta
= \frac{\sigma}{2\epsilon_0}\int_{\theta_0/2}^{\pi/2}\cos u du = \frac{\sigma}{2\epsilon_0}
\left(1 - \sin\left(\frac{\theta}{2}\right)\right).
\]

\item[13] Consider the potential due to a uniformly charged cube at one of its 
corners. Let the origin be at the diagonally opposite corner. Then the expression
for the potential is
\[
\phi(s) = \ke\int_0^s dx \int_0^s dy \int_0^s dz \frac{\rho}{\sqrt{x^2 + y^2 + z^2}}dxdydz.
\]
If the dimension of the cube is halved, the volume drops by a factor of $8$ and the
integrand by a factor of $1/2$ (from the denominator). Therefore,
\[
\phi\left(\frac{s}{2}\right) = \frac{1/8}{1/2}\phi(s) = \frac{\phi(s)}{4}
\]
so that
\[
8\phi\left(\frac{s}{2}\right) = 2\phi(s).
\]
Thus, the potential at the centre is twice as that at the corner.

\item[15] Electrostatic energy is
\[
U = \frac{\epsilon_0}{2}\int E^2 d^3r.
\]
Therefore, we need an expression for the electric field at various points in the
charge system. There is no electric field inside the smaller sphere because there
is no charge in it. Charge enclosed in the outer sphere is
\[
Q(R) = \rho \frac{4\pi}{3} (R^3 - a^3).
\]
At intermediate radii $a \le r \le R$, it is
\[
Q(r) = \rho \frac{4\pi}{3} (r^3 - a^3).
\]
Therefore, the electric field is
\[
E \times 4\pi r^2 = \frac{1}{\epsilon_0}Q(r) = \rho \frac{4\pi}{3\epsilon_0} (r^3 - a^3).
\]
so that
\[
E = \frac{\rho}{3\epsilon_0}\frac{r^3 - a^3}{r^2} \; a \le r \le R.
\]
For $r > R$, we have
\[
E = \ke\frac{Q(R)}{r^2}.
\]
Thus,
\[
\vec{E} = \begin{cases} 0 & \; 0 \le r < a \\
 \frac{\rho}{3\epsilon_0} \frac{(r^3 - a^3)}{r^2} \uv{r} & \; a \le r \le R \\
\ke\frac{Q(R)}{r^2} \uv{r} 
\end{cases}
\]
Therefore,
\[
U = \frac{\epsilon_0}{2}\left(0 + 
\iiint_a^R \frac{\rho^2}{9\epsilon_0^2}\frac{(r^3-a^3)^2}{r^4}r^2\sin\theta drd\theta d\phi +
\iiint_R^\infty \frac{Q^2}{16\pi^2\epsilon^2}\frac{1}{r^4}r^2\sin\theta drd\theta d\phi\right)
\]
The $\theta$ and $\phi$ integrals can be readily evaluated,
\[
U = \frac{4\pi\epsilon_0}{2}\left(
\int_a^R \frac{\rho^2}{9\epsilon_0^2}\frac{(r^3-a^3)^2}{r^2} dr +
\int_R^\infty \frac{Q^2}{16\pi^2\epsilon_0^2}\frac{1}{r^2} dr\right).
\]
After evaluating the integrals,
\[
U = \frac{4\pi\epsilon_0}{2}\left(\frac{\rho^2}{9\epsilon_0^2}\left(
\frac{R^5}{5} + \frac{9}{5}a^5 - a^3R^2 - \frac{a^6}{R}\right) + 
\frac{Q^2}{16\pi^2\epsilon_0^2}\frac{1}{R}\right).
\]
Now,
\[
\rho = \frac{3Q}{4\pi(R^3 - a^3)}
\]
implies that
\begin{eqnarray*}
U &=& \frac{4\pi\epsilon_0}{2}\left(\frac{Q^2}{16\pi^2\epsilon_0^2(R^3 - a^3)^2}\left(
\frac{R^5}{5} + \frac{9}{5}a^5 - a^3R^2 - \frac{a^6}{R}\right) + 
\frac{Q^2}{16\pi^2\epsilon_0^2}\frac{1}{R}\right) \\
 &=&  \frac{Q^2}{8\pi\epsilon_0R}\left(\frac{1}{(R^3 - a^3)^2}\left(
\frac{R^6}{5} + \frac{9}{5}a^5R - a^3R^3 - a^6\right) + 1\right) 
\end{eqnarray*}
If $a = 0$ then
\[
U = \frac{Q^2}{8\pi\epsilon_0R}\frac{6}{5} = \ke\frac{3}{5}\frac{Q^2}{R}.
\]
In order to find the limit $a \rightarrow R$. Since,
\[
\lim_{a \rightarrow R}\frac{R^6 + 9a^5R - 5a^3R^3 - 5a^6}{5(R^3 - a^3)^2} = 0,
\]
we see that
\[
U = \ke\frac{1}{2}\frac{Q^2}{R}.
\] 

\item[16] The energy of interaction between spheres is
\[
V_E = \frac{1}{2}\int d^2r \sigma_1(r)\varphi_2(r) + \frac{1}{2}\int d^2r \sigma_2(r)\varphi_1(r),
\]
where $\varphi_1(\varphi_2)$ is the potential due to sphere $1(2)$ on the surface
of sphere $2(1)$. Since the spheres are identical,
\[
V_E = \int d^2r \sigma_1(r)\varphi_2(r).
\]
Since the two spheres are well separated, the field on the surface of sphere $1$
is as if due to as point charge $Q$ located at the centre of sphere $2$, and 
\emph{vice versa}. Thus,
\[
\varphi_2 = \ke\frac{Q}{r}.
\]
where $r$ is the distance from $Q$. The $r$ in this equation is not the same as the
one in the expression for $V_E$. Now,
\[
\sigma_1 = \frac{Q}{4\pi r^2}
\]
so that
\[
V_E = \int d^2 r \frac{Q}{4\pi r^2} \ke \frac{Q}{\sqrt{r^2 + d^2 - 2rd\cos\theta}},
\]
where we aligned the $z$-axis along the line joining the centres of the two spheres.
Since the integration is over the surface of the sphere,
\[
V_E = \ke\frac{Q^2}{4\pi}\int_0^{2\pi}d\phi \int_0^\pi \frac{\sin\theta d\theta}{\sqrt{r^2 + d^2 - 2rd\cos\theta}},
\]
Let $u = r^2 + d^2 - 2rd\cos\theta$ so that $2rd\sin\theta d\theta = du$ and the 
limits of integration go from $(r - d)^2$ to $(r + d)^2$ and
\[
V_E = \ke \frac{Q^2}{2}\int_{(r - d)^2}^{(r + d)^2}\frac{1}{2rd}\frac{du}{\sqrt{u}} 
= \ke\frac{Q^2}{2}\frac{1}{rd}(d + r - d + r) = \ke\frac{Q^2}{d}.
\]
\end{enumerate}
\end{document}