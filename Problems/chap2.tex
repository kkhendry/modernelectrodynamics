\documentclass{article}
\usepackage{amsmath}
\newcommand{\un}{\hat{n}}
\title{The Maxwell Equations}\label{c2}
\author{Amey Joshi}
\date{02-Feb-2024}
\begin{document}
\maketitle
\begin{enumerate}
\item[(2.3)] The force between current loops.
\begin{enumerate}
\item[(a)] Note that,
\[
\nabla_1\left(\frac{1}{|\vec{r}_1 - \vec{r}_2|}\right) = -\frac{\vec{r}_1 - \vec{r}_2}{|\vec{r}_1 - \vec{r}_2|^3}
\]
so that
\begin{eqnarray*}
\oint_C d\vec{s}_1\cdot\frac{\vec{r}_1 - \vec{r}_2}{|\vec{r}_1 - \vec{r}_2|^3}
 &=& -\oint_C d\vec{s}_1\cdot\nabla_1\left(\frac{1}{|\vec{r}_1 - \vec{r}_2|}\right) \\
 &=& -\int_S d\vec{S}_1\cdot\nabla_1\times\left(\nabla_1\left(\frac{1}{|\vec{r}_1 - \vec{r}_2|}\right)\right)\\
 &=& 0.
\end{eqnarray*}

\item[(b)] We use the fact that
\[
\vec{B}_2(\vec{r}_1) = \frac{\mu_0}{4\pi}\oint_{C_2} I_2d\vec{s}_2 \times \frac{\vec{r}_1 - \vec{r}_2}{|\vec{r}_1 - \vec{r}_2|^3}
\]
in the expression
\begin{eqnarray*}
\vec{F} &=& \oint_{C_1}d\vec{s}_1 I_1 \times \vec{B}_2(\vec{r}_1) \\
 &=& \oint_{C_1}d\vec{s}_1I_1 \times \frac{\mu_0}{4\pi}\oint_{C_2} 
 	I_2d\vec{s}_2 \times \frac{\vec{r}_1 - \vec{r}_2}{|\vec{r}_1 - \vec{r}_2|^3} \\
 &=& \frac{\mu_0}{4\pi}\oint_{C_1}\oint_{C_2}d\vec{s}_1\times\left(d\vec{s}_s\times
        \frac{\vec{r}_1 - \vec{r}_2}{|\vec{r}_1 - \vec{r}_2|^3}\right)I_1I_2 \\
 &=& \frac{\mu_0}{4\pi}\oint_{C_1}\oint_{C_2}d\vec{s}_2\left(d\vec{s}_1\cdot
        \frac{\vec{r}_1 - \vec{r}_2}{|\vec{r}_1 - \vec{r}_2|^3}\right)I_1I_2  - \\
 & &  \frac{\mu_0}{4\pi}\oint_{C_1}\oint_{C_2}\frac{\vec{r}_1 - \vec{r}_2}{|\vec{r}_1 - \vec{r}_2|^3}
      d\vec{s}_1 \cdot d\vec{s}_2 I_1I_2
\end{eqnarray*}
The currents are constant in their loops. Therefore, the quantities $I_1, I_2$ 
can be pulled out of the integrals.
\begin{eqnarray*}
\vec{F} &=& \frac{\mu_0}{4\pi}I_1I_2\oint_{C_2}d\vec{s}_2\oint_{C_1}d\vec{s}_1\cdot
        \frac{\vec{r}_1 - \vec{r}_2}{|\vec{r}_1 - \vec{r}_2|^3} \\
 & & -\frac{\mu_0}{4\pi}\oint_{C_1}d\vec{s}_1I_1\oint_{C_2}d\vec{s}_2 I_2\frac{\vec{r}_1 - \vec{r}_2}{|\vec{r}_1 - \vec{r}_2|^3}
\end{eqnarray*}
The integral in the first term is zero according to part (a). The result
follows immediately from part (b).
\end{enumerate}

\item[(2.4)] If $\nabla\times\vec{B} = \mu_0\vec{j}$ then 
$\nabla\cdot\nabla\times\vec{B} = \mu_0\nabla\cdot\vec{j} \Rightarrow 
\nabla\cdot\vec{j} = 0$. This is true only for steady currents for which 
$\partial_t\rho = 0$.

\item[(2.5)] A charge is placed at $(x, y)$ in the plane. At $t = 0$ a magnetic field $\vec{B} = \Phi\delta(x)\delta(y)$ is
turned on. We want to find the angular momentum of the charge. A changing magnetic field induces an electric field. It is hard
to compute it from the differential form of Faraday's law. However, much like Gauss's law, it is convenient to use the integral
form,
\[
\oint_C \vec{E}\cdot d\vec{l} = -\frac{d}{dt}\int_S\vec{B}\cdot\hat{n}da.
\]
Since $\vec{B}$ points along the $z$ axis, by Lenz's law, $\vec{E}$ is azimuthal. If we take $C$ to be a circle of radius
$r$ centred at $(x, y)$, then the integral evaluates to $2\pi r E$. The integral on the right hand side is easy to evaluate
because of the $\delta$-function. It is just $\Phi$. Thus, we have
\[
E = -\frac{1}{2\pi r}\frac{d\Phi}{dt}.
\]
Being azimuthal, we can write the electric field vector as
\begin{equation}
\vec{E} = -\frac{1}{2\pi r}\frac{d\Phi}{dt}\hat{\phi}.
\end{equation}
The force acting on the charge is $q(\vec{E} + \vec{v} \times \vec{B})$ and the torque acting on it is
\[
\vec{N} = q\vec{r} \times \vec{E} + q\vec{r} \times (\vec{v} \times \vec{B})
= q\vec{r} \times \vec{E} + q\vec{v}(\vec{r}\cdot\vec{B}) - \vec{B}(\vec{r}\cdot\vec{v}).
\]
Since the motion is confined to the plane, ($\vec{E}$ being in the plane), $\vec{r}\cdot\vec{B} = 0$. The effect of a magnetic
field is to turn trajectory of the particle. Therefore, $\vec{v}$ too will be azimuthal, as a result $\vec{r}\cdot\vec{v} = 0$.
Thus, we have
\begin{equation}
\frac{d\vec{L}}{dt} = \vec{N} = q\vec{r}\times\vec{E} = -\frac{1}{2\pi}\frac{d\Phi}{dt}\hat{z},
\end{equation}
so that
\begin{equation}
\frac{d}{dt}\left(\vec{L} + \frac{\Phi}{2\pi}\right) = 0.
\end{equation}

\item[(2.7)] The source-free Maxwell equations are
\begin{eqnarray*}
\nabla\cdot\vec{E} &=& 0 \\
\nabla\cdot\vec{B} &=& 0 \\
\nabla\times\vec{E} &=& -\frac{\partial\vec{B}}{\partial t} \\
\nabla\times\vec{B} &=& \mu_0\epsilon_0\frac{\partial\vec{E}}{\partial t}
\end{eqnarray*}
Since
\[
c = \frac{1}{\sqrt{\epsilon_0\mu_0}},
\]
Ampere's law can be written as
\[
\nabla\times\vec{B} = \frac{1}{c^2}\frac{\partial\vec{E}}{\partial t}.
\]
Since $\vec{E}^\prime = \vec{E}\cos\theta + c\vec{B}\sin\theta$, we immediately infer that $\nabla\cdot\vec{E}^\prime=0$. Now
\[
\nabla\times\vec{E}^\prime = -\frac{\partial\vec{B}}{\partial t}\cos\theta + \frac{1}{c}\frac{\partial\vec{E}}{\partial t}
\sin\theta = -\frac{\partial\vec{B}^\prime}{\partial t}.
\]
Since $c\vec{B}^\prime = -\vec{E}\sin\theta + c\vec{B}\cos\theta$, we immediately get $\nabla\cdot\vec{B}^\prime = 0$.
\[
\nabla\times \vec{B}^\prime = -\frac{\sin\theta}{c}\nabla\times\vec{E} + \nabla\times\vec{B}\cos\theta = 
\frac{\sin\theta}{c}\frac{\partial\vec{B}}{\partial t} + \frac{1}{c^2}\frac{\partial\vec{E}}{\partial t}\cos\theta
\]
or
\[
\nabla\times \vec{B}^\prime = \frac{1}{c^2}\frac{\partial}{\partial t}(\vec{E}\cos\theta + c\vec{B}\sin\theta)
= \frac{1}{c^2}\frac{\partial B^\prime}{\partial t}.
\]
The parameter $\theta$ is the angle of polarization. If $\vec{E}$ and $\vec{B}$ are solutions of Maxwell equations in free
space then so are the corresponding primed fields.

\item[(2.8)] Consider Ampere's law in vacuum, $\nabla\times\vec{H} = \vec{J}_f + \partial_t\vec{D}$. In terms of $\vec{E}$
and $\vec{B}$, it reads $\nabla\times\vec{B} = \mu_0\vec{J}_j + c^{-2}\partial_t\vec{E}$. In integral form,
\[
\oint_C\vec{B}\cdot d\vec{l} = \mu_0\int_S\vec{J}_f\cdot\un da + \frac{1}{c^2}\frac{\partial}{\partial t}\int_S\vec{E}\cdot\un da.
\]
If the loop $C$ is entirely outside the infinite solenoid, it will not enclose any current density. So if $\vec{E}$ is
independent of time then $\vec{B}$ will vanish outside the solenoid. We are given that the surface current density varies
as $\vec{K} = K_0(t/\tau)$. It has no spatial dependence and therefore its divergence will be zero. As a result, the current
is steady and the electric field is independent of time.

Since the magnetic field is confined only to the interior of the solenoid, by symmetry it will be along the axis of the 
solenoid. In the case of surface currents, the second term will be replaced by
\[
\oint_C \vec{K}\cdot d\vec{l}.
\]
Thus, we have
\[
\oint_C\vec{B}\cdot d\vec{l} = \mu_0\oint_C \vec{K}\cdot d\vec{l} \Rightarrow 2\pi rB = 2\pi\mu_0 r K
\]
or
\begin{equation}
\vec{B} = \mu_0 K \hat{z}.
\end{equation}

In order to find the electric field, we use Faraday's law in integral form,
\[
\oint_C \vec{E}\cdot d\vec{l} = -\frac{d}{dt}\int_S \vec{B}\cdot\un da.
\]
If $C$ is entirely inside the solenoid, 
\[
2\pi r E = -\frac{d}{dt}\mu_0 K \pi r^2 \Rightarrow E = -\frac{\mu_0}{2}r \frac{dK}{dt}.
\]
If $C$ is outside the solenoid,
\[
2\pi r E = -\frac{d}{dt}\mu_0 K \pi R^2 \Rightarrow E = -\frac{\mu_0}{2}\frac{R^2}{r} \frac{dK}{dt},
\]
where $R$ is the radius of the solenoid.

\item[(2.14)] Given that
\[
\varphi(\vec{r}) = \frac{q}{4\pi\epsilon_0}\frac{1}{r^{1+\eta}}.
\]
Let the origin be chosen to coincide with the centre of the sphere whose radius is $R$. We can take advantage of the
spherical symmetry if we also align the $z$-axis along the line joining $O$ and $P$ where we want to find the potential. If $P$
is inside the sphere and if $OP = s$ then $PR = r = \sqrt{R^2 + s^2 - 2sR\cos\theta}$ where $\theta$ is the polar coordinate of
the point $Q$ on the sphere such that $\angle OPQ = 90^\circ$. Then
\begin{eqnarray*}
\varphi(P) &=& \frac{1}{4\pi\epsilon_0}\int_0^{2\pi}d\phi \int_0^\pi d\theta \sin\theta\frac{\sigma R^2}{r^{1+\eta}} \\
 &=& \frac{\sigma R^2}{4\pi\epsilon_0} 2\pi \int_0^\pi \frac{d\theta \sin\theta}{(R^2 + s^2 - 2sR\cos\theta)^{(1 + \eta)/2}} \\
 &=& \frac{\sigma R^2}{2\epsilon_0} \int_0^\pi \frac{\sin\theta}{(R^2 + s^2 - 2sR\cos\theta)^{(1 + \eta)/2}}d\theta
\end{eqnarray*}
Put $x = \cos\theta$ so that
\[
\varphi(P) = \frac{\sigma R^2}{2\epsilon_0}\int_{-1}^1 \frac{dx}{(R^2 + s^2 - 2sRx)^{(1 + \eta)/2}}.
\]
Put $u = R^2 + s^2 - 2Rsx$ so that
\[
\varphi(P) = \frac{\sigma R}{4s\epsilon_0}\int_{(R-s)^2}^{(R+s)^2} u^{-(1 + \eta)/2}du = 
\frac{\sigma R}{4s\epsilon_0}\frac{2}{1 - \eta}u^{(1 - \eta)/2}\Big|_{(R-s)^2}^{(R+s)^2}
\]
or
\[
\varphi(P) = \frac{\sigma R}{2s\epsilon_0(1 - \eta)}\left(\frac{1}{(R+s)^{(\eta - 1)}} - \frac{1}{(R-s)^{(\eta - 1)}}\right).
\]
\end{enumerate}
\end{document}